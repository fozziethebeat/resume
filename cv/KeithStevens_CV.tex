%%%%%%%%%%%%%%%%%%%%%%%%%%%%%%%%%%%%%%%%%%%%%%%%%%%%%%%%%%%%%%%%%%%%%%%%
%%%%%%%%%%%%%%%%%%%%%% Simple LaTeX CV Template %%%%%%%%%%%%%%%%%%%%%%%%
%%%%%%%%%%%%%%%%%%%%%%%%%%%%%%%%%%%%%%%%%%%%%%%%%%%%%%%%%%%%%%%%%%%%%%%%

%%%%%%%%%%%%%%%%%%%%%%%%%%%%%%%%%%%%%%%%%%%%%%%%%%%%%%%%%%%%%%%%%%%%%%%%
%% NOTE: If you find that it says                                     %%
%%                                                                    %%
%%                           1 of ??                                  %%
%%                                                                    %%
%% at the bottom of your first page, this means that the AUX file     %%
%% was not available when you ran LaTeX on this source. Simply RERUN  %%
%% LaTeX to get the ``??'' replaced with the number of the last page  %%
%% of the document. The AUX file will be generated on the first run   %%
%% of LaTeX and used on the second run to fill in all of the          %%
%% references.                                                        %%
%%%%%%%%%%%%%%%%%%%%%%%%%%%%%%%%%%%%%%%%%%%%%%%%%%%%%%%%%%%%%%%%%%%%%%%%

%%%%%%%%%%%%%%%%%%%%%%%%%%%% Document Setup %%%%%%%%%%%%%%%%%%%%%%%%%%%%

% Don't like 10pt? Try 11pt or 12pt
\documentclass[10pt]{article}

% This is a helpful package that puts math inside length specifications
\usepackage{calc}

% Simpler bibsection for CV sections
% (thanks to natbib for inspiration)
\makeatletter
\newlength{\bibhang}
\setlength{\bibhang}{1em}
\newlength{\bibsep}
 {\@listi \global\bibsep\itemsep \global\advance\bibsep by\parsep}
\newenvironment{bibsection}%
        {\begin{list}{}{%
       \setlength{\leftmargin}{\bibhang}%
       \setlength{\itemindent}{-\leftmargin}%
       \setlength{\itemsep}{\bibsep}%
       \setlength{\parsep}{\z@}%
        \setlength{\partopsep}{0pt}%
        \setlength{\topsep}{0pt}}}
        {\end{list}\vspace{-.6\baselineskip}}
\makeatother

% Layout: Puts the section titles on left side of page
\reversemarginpar

%
%         PAPER SIZE, PAGE NUMBER, AND DOCUMENT LAYOUT NOTES:
%
% The next \usepackage line changes the layout for CV style section
% headings as marginal notes. It also sets up the paper size as either
% letter or A4. By default, letter was used. If A4 paper is desired,
% comment out the letterpaper lines and uncomment the a4paper lines.
%
% As you can see, the margin widths and section title widths can be
% easily adjusted.
%
% ALSO: Notice that the includefoot option can be commented OUT in order
% to put the PAGE NUMBER *IN* the bottom margin. This will make the
% effective text area larger.
%
% IF YOU WISH TO REMOVE THE ``of LASTPAGE'' next to each page number,
% see the note about the +LP and -LP lines below. Comment out the +LP
% and uncomment the -LP.
%
% IF YOU WISH TO REMOVE PAGE NUMBERS, be sure that the includefoot line
% is uncommented and ALSO uncomment the \pagestyle{empty} a few lines
% below.
%

%% Use these lines for letter-sized paper
\usepackage[paper=letterpaper,
            %includefoot, % Uncomment to put page number above margin
            marginparwidth=1.2in,     % Length of section titles
            marginparsep=.05in,       % Space between titles and text
            margin=1in,               % 1 inch margins
            includemp]{geometry}

%% Use these lines for A4-sized paper
%\usepackage[paper=a4paper,
%            %includefoot, % Uncomment to put page number above margin
%            marginparwidth=30.5mm,    % Length of section titles
%            marginparsep=1.5mm,       % Space between titles and text
%            margin=25mm,              % 25mm margins
%            includemp]{geometry}

%% More layout: Get rid of indenting throughout entire document
\setlength{\parindent}{0in}

\usepackage[shortlabels]{enumitem}

%% Reference the last page in the page number
%
% NOTE: comment the +LP line and uncomment the -LP line to have page
%       numbers without the ``of ##'' last page reference)
%
% NOTE: uncomment the \pagestyle{empty} line to get rid of all page
%       numbers (make sure includefoot is commented out above)
%
\usepackage{fancyhdr,lastpage}
\pagestyle{fancy}
%\pagestyle{empty}      % Uncomment this to get rid of page numbers
\fancyhf{}\renewcommand{\headrulewidth}{0pt}
\fancyfootoffset{\marginparsep+\marginparwidth}
\newlength{\footpageshift}
\setlength{\footpageshift}
          {0.5\textwidth+0.5\marginparsep+0.5\marginparwidth-2in}
\lfoot{\hspace{\footpageshift}%
       \parbox{4in}{\, \hfill %
                    \arabic{page} of \protect\pageref*{LastPage} % +LP
%                    \arabic{page}                               % -LP
                    \hfill \,}}

% Finally, give us PDF bookmarks
\usepackage{color,hyperref}
\definecolor{darkblue}{rgb}{0.0,0.0,0.3}
\hypersetup{colorlinks,breaklinks,
            linkcolor=darkblue,urlcolor=darkblue,
            anchorcolor=darkblue,citecolor=darkblue}

%%%%%%%%%%%%%%%%%%%%%%%% End Document Setup %%%%%%%%%%%%%%%%%%%%%%%%%%%%


%%%%%%%%%%%%%%%%%%%%%%%%%%% Helper Commands %%%%%%%%%%%%%%%%%%%%%%%%%%%%

% The title (name) with a horizontal rule under it
% (optional argument typesets an object right-justified across from name
%  as well)
%
% Usage: \makeheading{name}
%        OR
%        \makeheading[right_object]{name}
%
% Place at top of document. It should be the first thing.
% If ``right_object'' is provided in the square-braced optional
% argument, it will be right justified on the same line as ``name'' at
% the top of the CV. For example:
%
%       \makeheading[\emph{Curriculum vitae}]{Your Name}
%
% will put an emphasized ``Curriculum vitae'' at the top of the document
% as a title. Likewise, a picture could be included:
%
%   \makeheading[\includegraphics[height=1.5in]{my_picutre}]{Your Name}
%
% the picture will be flush right across from the name.
\newcommand{\makeheading}[2][]%
        {\hspace*{-\marginparsep minus \marginparwidth}%
         \begin{minipage}[t]{\textwidth+\marginparwidth+\marginparsep}%
             {\large \bfseries #2 \hfill #1}\\[-0.15\baselineskip]%
                 \rule{\columnwidth}{1pt}%
         \end{minipage}}

% The section headings
%
% Usage: \section{section name}
\renewcommand{\section}[1]{\pagebreak[3]%
    \hyphenpenalty=10000%
    \vspace{1.3\baselineskip}%
    \phantomsection\addcontentsline{toc}{section}{#1}%
    \noindent\llap{\scshape\smash{\parbox[t]{\marginparwidth}{\raggedright #1}}}%
    \vspace{-\baselineskip}\par}

% An itemize-style list with lots of space between items
\newenvironment{outerlist}[1][\enskip\textbullet]%
        {\begin{itemize}[#1,leftmargin=*]}{\end{itemize}%
         \vspace{-.6\baselineskip}}

% An environment IDENTICAL to outerlist that has better pre-list spacing
% when used as the first thing in a \section
\newenvironment{lonelist}[1][\enskip\textbullet]%
        {\begin{list}{#1}{%
        \setlength{\partopsep}{0pt}%
        \setlength{\topsep}{0pt}}}
        {\end{list}\vspace{-.6\baselineskip}}

% An itemize-style list with little space between items
\newenvironment{innerlist}[1][\enskip\textbullet]%
        {\begin{itemize}[#1,leftmargin=*,parsep=0pt,itemsep=0pt,topsep=0pt,partopsep=0pt]}
        {\end{itemize}}

% An environment IDENTICAL to innerlist that has better pre-list spacing
% when used as the first thing in a \section
\newenvironment{loneinnerlist}[1][\enskip\textbullet]%
        {\begin{itemize}[#1,leftmargin=*,parsep=0pt,itemsep=0pt,topsep=0pt,partopsep=0pt]}
        {\end{itemize}\vspace{-.6\baselineskip}}

% To add some paragraph space between lines.
% This also tells LaTeX to preferably break a page on one of these gaps
% if there is a needed pagebreak nearby.
\newcommand{\blankline}{\quad\pagebreak[3]}
\newcommand{\halfblankline}{\quad\vspace{-0.5\baselineskip}\pagebreak[3]}

% Uses hyperref to link DOI
\newcommand\doilink[1]{\href{http://dx.doi.org/#1}{#1}}
\newcommand\doi[1]{doi:\doilink{#1}}

% For \url{SOME_URL}, links SOME_URL to the url SOME_URL
\providecommand*\url[1]{\href{#1}{#1}}
% Same as above, but pretty-prints SOME_URL in teletype fixed-width font
\renewcommand*\url[1]{\href{#1}{\texttt{#1}}}

% For \email{ADDRESS}, links ADDRESS to the url mailto:ADDRESS
\providecommand*\email[1]{\href{mailto:#1}{#1}}
% Same as above, but pretty-prints ADDRESS in teletype fixed-width font
%\renewcommand*\email[1]{\href{mailto:#1}{\texttt{#1}}}

%\providecommand\BibTeX{{\rm B\kern-.05em{\sc i\kern-.025em b}\kern-.08em
%    T\kern-.1667em\lower.7ex\hbox{E}\kern-.125emX}}
%\providecommand\BibTeX{{\rm B\kern-.05em{\sc i\kern-.025em b}\kern-.08em
%    \TeX}}
\providecommand\BibTeX{{B\kern-.05em{\sc i\kern-.025em b}\kern-.08em
    \TeX}}
\providecommand\Matlab{\textsc{Matlab}}

%%%%%%%%%%%%%%%%%%%%%%%% End Helper Commands %%%%%%%%%%%%%%%%%%%%%%%%%%%

%%%%%%%%%%%%%%%%%%%%%%%%% Begin CV Document %%%%%%%%%%%%%%%%%%%%%%%%%%%%

\begin{document}
\makeheading{Keith D. Stevens}

\section{Contact Information}

% NOTE: Mind where the & separators and \\ breaks are in the following
%       table.
%
% ALSO: \rcollength is the width of the right column of the table
%       (adjust it to your liking; default is 1.85in).
%
\newlength{\rcollength}\setlength{\rcollength}{1.85in}%
%
\begin{tabular}[t]{@{}p{\textwidth-\rcollength}p{\rcollength}}
\href{http://iscr.llnl.gov/}%
     {Institute for Scientific Computing Research} & \textit{Mobile:} +1-805-433-2295\\
\href{http://www.llnl.gov/}{Lawrence Livermore National Lab}
7000 East Ave,       & \textit{E-mail:} \email{kstevens@cs.ucla.edu}\\
Livermore, CA 94551  & \textit{WWW:}
\href{http://kracken.cs.ucla.edu/}{kracken.cs.ucla.edu}\\
\end{tabular}

\section{Objective}

Placement in a development and/or research position centered around distributed
semantics and unsupervised learning, along with applying current theories on
these topics to web search or other human oriented data exploration.

\section{Research Interests}

My research focuses on the unsupervised acquisition of word senses and the
automated evaluation of related models.  Word sense acquisition, or Word Sense
Induction (WSI), attempts to use contextual information to learn distinct word
senses as they appear in large corpora.  My initial work centered around
developing a simple framework for building and applying WSI models.  My work now
focuses on finding automated intrinsic and extrinsic evaluation metrics that
explain how to improve these models.


\section{Education}

\href{http://www.ucla.edu/}{\textbf{University of California, Los Angeles}},
Los Angeles, California USA
\begin{outerlist}

\item[] Ph.D.,
        \href{http://www.cs.ucla.edu/}
             {Computer Science},
             In Progress
        \begin{innerlist}
        \item Thesis Topic: \emph{Intrinsic and Extrinsic evaluation frameworks
        for Word Sense Induction models}
        \item Adviser:
              \href{http://www.cs.ucla.edu/~dyer/}
                   {Professor Michael G.~Dyer}
        \item Area of Study: Computational Linguistics
        \end{innerlist}

\item[] M.S.,
        \href{http://www.cs.ucla.edu/}
             {Computer Science}, October 2011
        \begin{innerlist}
        \item Thesis Topic: \emph{Extending Word Sense Induction with waiting
        and depending}
        \item Adviser:
              \href{http://www.cs.ucla.edu/~dyer/}
                   {Professor Michael G.~Dyer}
        \item Area of Study: Computational Linguistics
        \end{innerlist}

\item[] B.S.,
        \href{http://www.cs.ucla.edu/}
             {Computer Science}, December 2007
        \begin{innerlist}
          \item Computer Science
          \item Minor in Mathematics
        \end{innerlist}

\end{outerlist}

\section{Research Experience}

\href{http://www.llnl.gov/}{\textbf{Lawrence Livermore National Lab}},
Livermore, California USA
\begin{outerlist}

\item[] \textbf{Word Sense and Topic Model Evaluation} \\
        June 2011 - \textit{present} \\
        Advisor: \textit{David Buttler} \\

        I am currently evaluating three ``topic models'' with two recently
        developed coherence metrics that have been closely associated with human
        judgements.  Our current results indicate that the coherence metrics,
        originally designed for Latent Dirichlet Allocation, can be applied to
        other techniques such as Non-negative Matrix Factorization and Singular
        Value Decomposition, as seen in Latent Semantic Analysis).  Furthermore,
        we are evaluating new aggregate metrics for comparing entire models,
        rather than just individual topics. \\

        Based on the above research, I am beginning to investigate the
        applicability of these coherence metrics as an intrinsic evaluation of
        Word Sense Induction (WSI) models.  Current evaluation strategies for
        WSI focus on costly human judgements and do not provide a method for
        improving the models.  Along these lines, I am also Consensus Clustering
        as an evaluation of the feature space used in WSI models and the
        stability of clustering models used to differentiate word senses. \\

\item[] \textbf{Ontology Expansion} \\
        June 2010 - September 2010 \\
        Advisor: \textit{David Buttler} \\

        With David Buttler, I have investigated methods of automatically
        enhancing existing concept hierarchies for the purpose of organizing
        unstructured text documents. This work is in coordination with my work
        in Distributional Semantics and will soon be used by several information
        retrieval tasks at the national lab. \\

\end{outerlist}

\href{http://www.cs.ucla.edu/}{\textbf{University of California, Los Angeles}},
Los Angeles, California USA
\begin{outerlist}

\item[] \textbf{Distributional Semantics} \\
        September 2008 - June 2011 \\
        Advisor: \textit{Michael G.~Dyer} \\

        Distributional semantics assumes that the semantics of words can be
        defined by the contexts in which they occur, an idea proposed by Firth
        in 1957. The learned semantics have been shown to approximate semantic
        judgements made by humans on psychological and standardized tests. My
        research in this field has three main goals. \\

        First, models for distributional semantics have been developed in a
        variety of fields, such as Computational Linguistics, Cognitive Science,
        and Artificial Intelligence. In order to permit accurate evaluation of
        distinct models, a common framework is needed for their implementation
        and evaluation. To this end, I have collaborated with David Jurgens to
        develop the S-Space Package, a java based framework for implementing,
        testing, and extending distributional models. \\

        Second, current distributional models are incapable of distinguishing
        between distinct meanings for a single word, for example, “cat” can
        refer to feline pets, large felines found in the wild, or a brand of
        construction vehicles. To learn these differences, I combine
        distributional models with clustering techniques. Initial research in
        this field, often known as Word Sense Induction, has already been shown
        to perform well according to one measure in a coordinated semantic task,
        SemEval, and to be applicable towards detecting the occurrence of news
        worthy events. \\

        Lastly, current distributional models lack organization. Word semantics
        are unorga- nized and disconnected from each other. In contrast, concept
        hierarchies provide a rich set of relationships between the semantics
        associated with many words. I investigate methods of combining
        distributional models and Word Sense Induction models for the purpose of
        automatically building concept hierarchies. For this, I focus on two
        approaches: extending existing hierarchies by enhancing current
        semi-supervised tech- niques and inferring hierarchies automatically
        from word semantics learned through only distributional methods.

\end{outerlist}

% Add a little space to nudge next ``Conference Publications'' marginpar
% down to make room for tall ``Submitted Journal Publications''
% marginpar. If there are enough submitted journal publications, this
% space will not be needed (and should be removed).
\vspace{0.1in}

\section{Publications}

\textbf{Referred} \\

\begin{bibsection}
  \item Keith Stevens, Terry Huang and David Buttler, ``The C-Cat Wordnet
        Package: An Open Source Package for modifying and applying Wordnet.''
        \emph{Systems Demonstration of the Global Wordnet Conference 2012},
        2012.

  \item David Jurgens and Keith Stevens, ``Measuring the Impact of Sense
        Similarity on Word Sense Induction.'' \emph{Proceedings of the EMNLP 2011
        Workshop on Unsupervised Learning in NLP}, 2011. ACL.

  \item David Jurgens and Keith Stevens. ``Capturing Nonlinear Structure in Word
        Spaces Through Dimensionality Reduction.'' \emph{Proceedings of the ACL 2010
        Workshop on GEometrical Models of Natural Language Semantics}, 2010.

  \item David Jurgens and Keith Stevens. ``HERMIT: Flexible Clustering for the
        SemEval-2 WSI Task.'' \emph{Proceedings of the ACL 2010 SemEval-2010
        Workshop}, 2010.

  \item David Jurgens and Keith Stevens. ``The S-Space Package: An Open-Source
        Framework for Word Space Algorithms.'' \emph{Proceedings of the ACL 2010
        System Demonstrations}, 2010.

  \item David Jurgens and Keith Stevens. ``Event Detection in Blogs using
        Temporal Random Indexing.'' \emph{In Proceedings of the International
        Workshop on Events on Emerging Text Types} (eETTs), pages 21-28, 2009. \\

\end{bibsection} 

\textbf{Manuscripts} \\

\begin{bibsection}
  \item David Jurgens and Keith Stevens. ``Word Ordering with Reduced
        Dimensionality for Sense Induction.'' Technical Report 090020,
        University of California Los Angeles, 2009.

  \item Keith Stevens. ``Extending Word Sense Induction with waiting and
        depending.'' Masters Thesis, University of California Los Angeles,
        2010. \\

\end{bibsection} 

\textbf{In Progress} \\

\begin{bibsection}
  \item Keith Stevens and Others. ``Applying Topic Coherence metrics over many models and
        many topics.''

  \item Keith Stevens and Others. ``Evaluation Induced Word Senses with Topic
        Coherence Metrics.''

  \item Keith Stevens and Others. ``Understanding Word Sense Induction with
        Consensus Clustering.'' \\

\end{bibsection}

\section{Teaching Experience}

\href{http://www..ucla.edu}{\textbf{University of California, Los Angeles}},
Los Angeles, California USA
\begin{outerlist}

\item[] \textbf{Computer Science 32 - Introduction to Computer Science II } \\
        Teaching Assistant, \textit{Winter 2009} \textit{Spring 2009} \\

        This course introduces students to basic object oriented design a
        fundamental data structures.  Topics include C++ classes, C++ templates,
        pointer management, trees and traversal methods, inheritance, and
        recursion.  As the teaching assistant, I was responsible for teaching a
        weekly 2 hour discussion section that reviewed class lectures and broke
        down class assignments.

\item[] \textbf{Computer Science 35L - Software Construction Laboratory} \\
        Teaching Assistant, \textit{Fall 2008} \textit{Fall 2009} \\

        This course introduces students to fundamental concepts for developing
        software in a unix environment.  Topics include Bash and Python
        scripting, The GNU debugger, basic C programming, version control,
        network security, and basic operating system tools.  As the teaching
        assistant, I was responsible for all lectures and lab assignments.

\item[] \textbf{Computer Science 111 - Introduction to Operating System
                                       Principles} \\
        Teaching Associate, \textit{Spring 2010} \textit{Fall 2010} \\

        This course introduces students to the basic theories behind modern
        operating systems, with a focus on theories behind the Linux Kernel.
        Topics include concurrency, device management, file systems, memory
        management, process and thread management, Remote Procedure Calls,
        scheduling, security and protection, and shell design.  As the teaching
        assistant, I applied theories taught in class to a six lab assignments
        that covered virtual memory, process management, kernel modules, shell
        design, file systems, and remote applications.  

\item[] \textbf{Computer Science 132 - Modern Compiler Construction} \\
        Teaching Associate, \textit{Winter 2011} \\

        This course introduces students to the theories and practices behind
        compiler design.  Students learn topics focusing on parsing, type
        checking, register allocation, and virtual machines. As the teaching
        assistant, I applied theories taught in lectures to practical examples
        of compilers and assisted students with lab assignments, along with
        grading of all lab assignments.
        
\item[] \textbf{Engineering 183 - Engineering and Society} \\
        Teaching Associate, \textit{Winter 2011} \\

        This course introduces students to professional and ethical
        considerations experienced in engineering positions. 
        Emphasis is placed on the impact of technology on society and the
        development of moral and ethical values. This course also serves as the
        technical writing course for the School of Engineering. Lectures cover
        ethical topics, while teaching assistant-led sections cover technical
        writing. \\

        I taught a weekly three-hour discussion on technical writing and ethics.
        As the teaching assistant, I covered all writing requirements for
        students and provided a series of writing exercises designed to improve
        their ability to concisely describe, evaluate, and solve ethical
        dilemmas in Engineering. \\

\end{outerlist}

\section{Professional Experience}

\href{http://www.llnl.gov}{\textbf{Lawrence Livermore National Lab}}, 
Livermore, CA
\begin{outerlist}

\item[] \textit{Student Research Intern},
        \textbf{June 2010 to September 2010}\\
        Mentor: \textit{David Buttler} \\

        \begin{innerlist}
        \item Implemented an existing taxonomy expansion algorithm for a highly
              distributed framework based on Hadoop and HBase
        \item Implemented several methods for condensing an existing taxonomy
        \item Evaluateedseveral methods for integrating taxonomy induction and
              condensation techniques
        \item Evaluated several methods for enhancing taxonomy expansion
              and condensation methods such that the resulting taxonomy is
              customized for a particular domain of knowledge
        \item Developed a new, java based interface for the widely used WordNet
              taxonomy \\
       \end{innerlist}
\end{outerlist}

\href{http://www.google.com}{\textbf{Google, Inc.}}, 
Kirkland, WA
\begin{outerlist}

\item[] \textit{Software Engineering Intern},
        \textbf{June 2009 to September 2009}\\
        Mentor: \textit{Zhen Lin} \\

        \begin{innerlist}
        \item Enhanced a document selection system with the addition of semantic
              analysis.  This analysis provided a more detailed representation
              of the document, as it changed over several points in time. This
              analysis allowed the application to inform users of more relevant
              aspects of the page.
        \item Enhanced crawling framework for the document selection system,
              improving savings by up to 50\% usage of external resources.
        \item Restructured document selection system to be more scalable and
              abide by internal policies.
        \end{innerlist}

\item[] \textit{Software Engineering Intern},
        \textbf{January 2008 to September 2008}\\
        Mentor: \textit{Hizakazu Igarashi} \\

        \begin{innerlist}
        \item Built a highly scalable, stateless server for responding to a wide
              variety, and high volume, of requests.
        \item Developed several plugins for the designed server allowing access
              to a wide variety of data sources.
        \item Improved serving time of the server by addition of caching
              responses in memory
        \item Improved response time of a query formulation system with the use
              of a Trie, and integration as a plugin for the server.
        \item Heavily unittested, document, and debugged same server. \\
        \end{innerlist}

\end{outerlist}

\halfblankline

\href{http://nesl.ee.ucla.edu/}{\textbf{Networked and Embedded Systems Laboratory}}, 
Los Angeles, CA
\begin{outerlist}

\item[] \textit{Student Research Intern},
        \textbf{March 2007 to December 2007}\\
        Mentor: \textit{Professor Mani Srivastava} \\

        \begin{innerlist}
        \item Designed multiple modules for use on a Micaz/Mica2 sensor board
              using the SOS embedded operating system, which is based on a
              Linux architecture.
        \item Reduced the transmission rate of nodes by defining a time series
              representing sensor values.
        \item Designed a unit test framework for the SOS operating system.
        \end{innerlist}
\end{outerlist}

\section{Service}

Recent contributor to several open-source software projects, including:
\begin{innerlist}
    \item \href{http://vim-latex.sourceforge.net/}{Vim-LaTeX} suite
    \item \href{http://vimperator.org}{Vimperator} and
        \href{http://dactyl.sourceforge.net/pentadactyl/index}{Pentadactyl}
        Firefox extensions
    \item \href{http://git-scm.com}{Git} distributed version control
        system
    \item \href{http://www.selenic.com/mercurial/}{Mercurial} distributed version control
        system
    \item Personal projects archived at
        \url{http://hg.tedpavlic.com/}
\end{innerlist}

\halfblankline

Frequent contributor to \href{http://www.wikipedia.org/}{Wikipedia}.
%
\begin{innerlist}
    \item Significant contributions to articles on control theory,
        electronics, and signals and systems.
\end{innerlist}

\halfblankline

\href{http://www.osufirst.org/}{OSU FIRST Robotics Team},
\href{http://www.osu.edu}{The Ohio State University}, 2000--2004
\begin{innerlist}
\item Introduced middle school and high school students to science and
        technology by participating with them in national robotics
        competitions.
\item Led 2002 team to regional silver medal
        \href{http://www.firstwiki.org/Engineering_Inspiration_Award}
             {\emph{Engineering Inspiration Award}}.
\item \emph{Lead Team Mentor}, 2002--2004
\item \emph{Component Design Team Lead Mentor}, 2001--2002
\end{innerlist}

\halfblankline

Institute for Electrical and Electronics Engineers~(IEEE), Member,
2002--present
%
\begin{innerlist}
\item IEEE Control Systems Society (2004--present)
\item IEEE Computer Society (2009--present)
\item IEEE Intelligent Transportation Systems Society (2011--present)
\item IEEE Systems, Man, and Cybernetics Society (2011--present)
\item IEEE Robotics and Automation Society (2011--present)
\end{innerlist}

\halfblankline

Animal Behavior Society, Member, 2011--present

\halfblankline

Director of Computers,
\href{http://ec.osu.edu/}{Engineers' Council},
\href{http://www.osu.edu/}{The Ohio State University}, 2002

\halfblankline

\href{http://www.linuxvirtualserver.org/}
     {Linux Virtual Server Project}, 1999--2000
\begin{innerlist}
\item Early member of the team that formed the open-source project that
        is now an important load balancing solution for the Linux
        software platform.
\end{innerlist}

\halfblankline

\href{http://www.gcfn.org/}
     {Greater Columbus Free-Net}, 1995--1997
\begin{innerlist}
\item Provided technical support services.
\end{innerlist}

\halfblankline

CompuTeen Bulletin Board System, 1993--1995
\begin{innerlist}
\item Administrated dial-up bulletin board system.
\item Founded and administrated TeenLiNK, an international electronic
        mail network that spread through the United States, Canada, and
        Australia and delivered mail over a series of electronic dial-up
        drop offs.
\end{innerlist}


\section{Application Areas}

Autonomous and Unmanned Vehicles, Flexible Manufacturing Systems,
Distributed Power Generation, Intelligent Lighting, Power Demand
Response, Microgrids, Smart Grids

\section{Hardware and Software Skills}

Analog and Digital Electronics:
%
\begin{innerlist}
    \item Bipolar and FET implementations of continuous and switched
        amplifiers, modulators, converters, and filters

    \item Computer-Aided Design Tools: Cadence OrCAD, NI Multisim, SPICE, pst-circ
\end{innerlist}

\halfblankline

Embedded and Real-time Systems:
%
\begin{innerlist}
    \item Software and hardware development with several MCU and
        DSP platforms (e.g., Motorola MCU's, Texas Instruments DSP's, Atmel
        ATmega MCU's, Microchip PIC MCU's, and others)
\end{innerlist}

\halfblankline

Instrumentation, Control, Data Acquisition, Test, and Measurement:
%
\begin{innerlist}
    \item \href{http://www.dspaceinc.com/}{dSPACE} hardware (e.g.,
        RTI1104) and Control Desk software,
        \href{http://www.mathworks.com/products/simulink/}{Simulink},
        \href{http://www.ni.com/}{LabVIEW} and other
        \href{http://www.ni.com}{National Instruments}
        control and data acquisition hardware and software (e.g., MIO,
        SMIO, DSA, DMM, and others), Hewlett-Packard and Agilent
        bench-top equipment
\end{innerlist}

\halfblankline

Computer Programming:
%
\begin{innerlist}
    \item C, C$+$$+$, Java, JavaScript, NetLogo, Pascal, Perl, PHP,
        Lisp, UNIX shell scripting (including POSIX.2), GNU make,
        AppleScript, SQL, MySQL, \Matlab, Maple, Mathematica, and others
\end{innerlist}

\halfblankline

Version Control and Software Configuration Management:
%
\begin{innerlist}
    \item DVCS (Mercurial/MQ, Git/StGit), VCS (RCS, CVS, SVN, SCCS), and
        others
\end{innerlist}

\halfblankline

\href{http://www.mathworks.com/products/matlab/}{\Matlab} skill set:
%
\begin{innerlist}
    \item Linear algebra, Fourier transforms, Monte Carlo analysis,
        nonlinear numerical methods, polynomials, statistics,
        $N$-dimensional filters, visualization

    \item Toolboxes: communications, control system, filter design,
        genetic algorithm and direct search, signal processing, system
        identification
\end{innerlist}

\halfblankline

Software Verification:
%
\begin{innerlist}
    \item KeY, PRISM, KeYmaera
\end{innerlist}

\halfblankline

Information/Internet Technology:
%
\begin{innerlist}
    \item Networking (UDP, TCP, ARP, DNS, Dynamic
        routing), Services (Apache, SQL, MediaWiki, POP, IMAP, SMTP,
        application-specific daemon design)
\end{innerlist}

\halfblankline

Productivity Applications:
%
\begin{innerlist}
    \item \TeX{} (\LaTeX{}, \BibTeX{}, PSTricks), Vim,
        most common productivity packages (for Windows, OS X, and Linux
        platforms)
\end{innerlist}

\halfblankline

Operating Systems:
%
\begin{innerlist}
    \item Microsoft Windows family, Apple OS X, IBM OS/2, Linux, BSD,
        IRIX, AIX, Solaris, and other UNIX variants
\end{innerlist}

\section{Expertise}

Mathematics:
%
\begin{innerlist}
    \item Applied Mathematics, Real and Complex Analysis, Measure
        Theory, Differential Geometry, Game Theory, Graph Theory,
        Combinatorics
\end{innerlist}

\halfblankline

Control Theory and Engineering:
%
\begin{innerlist}
    \item Linear and Nonlinear Systems Theory, Feedback, Variable
        Structure Systems and Sliding Modes, Distributed and Intelligent
        Control, Dynamic Optimization, Bio-mimicry, Bio-inspiration,
        Hybrid and CyberPhysical Systems
\end{innerlist}

\halfblankline

Communications and Signal Processing:
%
\begin{innerlist}
    \item Probability, Random Variables, Stochastic Processes,
        Information Theory, Estimation, Networks
\end{innerlist}

\halfblankline

Computer Science and Engineering:
%
\begin{innerlist}
    \item Model Checking (automated, distributed, hybrid,
        probabilistic), Hybrid Automata, Software Verification,
        Component-Based Reusable Software
\end{innerlist}

\halfblankline

Natural Sciences (Biology, Neuroscience, Psychology, Anthropology):
%
\begin{innerlist}
    \item Behavioral Ecology, Foraging Theory, Altruism, Impulsiveness,
        Evolution
\end{innerlist}

\section{References Available to Contact}

\href
{http://www.ece.osu.edu/~passino/}
{\textbf{Dr.~Kevin M.~Passino}}
(e-mail:~\href{mailto:passino.1@osu.edu}{passino.1@osu.edu}; phone:~+1-614-312-2472)
%
\begin{innerlist}
    \item Professor,
        \href{http://www.ece.osu.edu/}{Electrical and Computer
        Engineering},
        \href{http://www.osu.edu/}{The Ohio State University}

    \item[$\diamond$] 205 Dreese Laboratories, 2015 Neil Ave., Columbus,
        OH  43210

    \item[$\star$] \emph{Dr.~Passino was my graduate adviser.}
\end{innerlist}

\halfblankline

\href
{http://www.cse.ohio-state.edu/~weide/}
{\textbf{Dr.~Bruce W.~Weide}}
(e-mail:~\href{mailto:weide.1@osu.edu}{weide.1@osu.edu}; phone:~+1-614-292-1517)
\begin{innerlist}
    \item Professor and Associate Chair,
        \href{http://www.cse.ohio-state.edu/}{Computer Science and
        Engineering}\\
        \href{http://www.osu.edu/}{The Ohio State University}

    \item[$\diamond$] 395 Dreese Laboratories, 2015 Neil Ave., Columbus,
        OH  43210

    \item[$\star$] \emph{Dr.~Weide is a co-PI on the NSF grant that
        funds my current postdoctoral position.}
\end{innerlist}

\halfblankline

\href
{http://hamilton-lab.wikidot.com/}
{\textbf{Dr.~Ian M.~Hamilton}}
(e-mail:~\href{mailto:hamilton.598@osu.edu}{hamilton.598@osu.edu}; phone:~+1-614-292-9147)
%
\begin{innerlist}
    \item Assistant Professor,
        \href{http://eeob.osu.edu/}{Evolution, Ecology, and Organismal Biology}
        and
        \href{http://www.math.ohio-state.edu/}{Mathematics}\\
        \href{http://www.osu.edu/}{The Ohio State University}

    \item[$\diamond$] 300 Aronoff Laboratory, 318 W.~12th Avenue,
        Columbus, OH  43210

    \item[$\star$] \emph{Dr.~Hamilton has been a valuable
        interdisciplinary resource to me.}
\end{innerlist}

\halfblankline

\href
{http://www.ece.osu.edu/~serrani/}
{\textbf{Dr.~Andrea Serrani}}
(e-mail:~\href{mailto:serrani.1@osu.edu}{serrani.1@osu.edu}; phone:~+1-614-292-4976)
%
\begin{innerlist}
    \item Associate Professor,
        \href{http://www.ece.osu.edu/}{Electrical and Computer Engineering}\\
        \href{http://www.osu.edu/}{The Ohio State University}

    \item[$\diamond$] 205 Dreese Laboratories, 2015 Neil Ave., Columbus,
        OH  43210

    \item[$\star$] \emph{Dr.~Serrani was a member of my doctoral
        committee.}
\end{innerlist}

\halfblankline

\href
{http://www.cse.ohio-state.edu/~paolo/}
{\textbf{Dr.~Paolo A.~G.~Sivilotti}}
(e-mail:~\href{mailto:sivilotti.1@osu.edu}{sivilotti.1@osu.edu}; phone: +1-614-292-5835)
\begin{innerlist}
    \item Associate Professor,
        \href{http://www.cse.ohio-state.edu/}{Computer Science and Engineering},
        \href{http://www.osu.edu/}{The Ohio State University}

    \item[$\diamond$] 395 Dreese Laboratories, 2015 Neil Ave., Columbus,
        OH  43210

    \item[$\star$] \emph{Dr.~Sivilotti is a co-PI on the NSF grant that
        funds my current postdoctoral position.}
\end{innerlist}

\halfblankline

\href
{http://feh.osu.edu/staff/view.html?UID=798}
{\textbf{Dr.~Richard J.~Freuler}}
(e-mail:~\href{mailto:freuler.1@osu.edu}{freuler.1@osu.edu}; phone: +1-614-688-0499)
\begin{innerlist}
    \item Professor of Practice,
        \href{http://mae.osu.edu/}{Mechanical and Aerospace Engineering}\\
        \href{http://www.osu.edu/}{The Ohio State University}

    \item[$\diamond$] 244 Hitchcock Hall, 2070 Neil Ave., Columbus, OH  43210

    \item[$\star$] \emph{Dr.~Freuler coordinates the Fundamentals of
        Engineering for Honors program in which I served as an
        instructor early in my academic career.}
\end{innerlist}

\halfblankline

\href
{http://mae.osu.edu/people/staab.1}
{\textbf{Dr.~George H.~Staab}}
(e-mail:~\href{mailto:staab.1@osu.edu}{staab.1@osu.edu}; phone: +1-614-292-7920)
\begin{innerlist}
    \item Associate Professor,
        \href{http://mae.osu.edu/}{Mechanical and Aerospace Engineering}\\
        \href{http://www.osu.edu/}{The Ohio State University}

    \item[$\diamond$] W192 Scott Laboratory, 201 W.~19th Ave., Columbus, OH  43210

    \item[$\star$] \emph{Dr.~Staab is the faculty adviser for the OSU
        FIRST robotics and engineering outreach group of which I was a
        four-year member and team leader.}
\end{innerlist}

\halfblankline

\textbf{Dr.~Clayton Daigle}
(e-mail:~\href{mailto:Clayton.Daigle@silabs.com}{Clayton.Daigle@silabs.com}; phone: +1-512-532-5935)
\begin{innerlist}
    \item Mixed-Signal Engineer,
        \href{http://www.silabs.com/}{Silicon Laboratories}, Austin, TX

    \item[$\star$] \emph{Dr.~Daigle was my direct supervisor when I
        worked for National Instruments as an analog hardware R\&D
        engineer.}
\end{innerlist}

\end{document}

%%%%%%%%%%%%%%%%%%%%%%%%%% End CV Document %%%%%%%%%%%%%%%%%%%%%%%%%%%%%

%----------------------------------------------------------------------%
% The following is copyright and licensing information for
% redistribution of this LaTeX source code; it also includes a liability
% statement. If this source code is not being redistributed to others,
% it may be omitted. It has no effect on the function of the above code.
%----------------------------------------------------------------------%
% Copyright (c) 2007, 2008, 2009, 2010, 2011 by Theodore P. Pavlic
%
% Unless otherwise expressly stated, this work is licensed under the
% Creative Commons Attribution-Noncommercial 3.0 United States License. To
% view a copy of this license, visit
% http://creativecommons.org/licenses/by-nc/3.0/us/ or send a letter to
% Creative Commons, 171 Second Street, Suite 300, San Francisco,
% California, 94105, USA.
%
% THE SOFTWARE IS PROVIDED "AS IS", WITHOUT WARRANTY OF ANY KIND, EXPRESS
% OR IMPLIED, INCLUDING BUT NOT LIMITED TO THE WARRANTIES OF
% MERCHANTABILITY, FITNESS FOR A PARTICULAR PURPOSE AND NONINFRINGEMENT.
% IN NO EVENT SHALL THE AUTHORS OR COPYRIGHT HOLDERS BE LIABLE FOR ANY
% CLAIM, DAMAGES OR OTHER LIABILITY, WHETHER IN AN ACTION OF CONTRACT,
% TORT OR OTHERWISE, ARISING FROM, OUT OF OR IN CONNECTION WITH THE
% SOFTWARE OR THE USE OR OTHER DEALINGS IN THE SOFTWARE.
%----------------------------------------------------------------------%
